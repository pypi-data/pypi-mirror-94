
According to IMPUTE2's documentation, the cross-validation tables are "based on
an internal cross-validation that is performed during each IMPUTE2 run. For
this analysis, the program masks the genotypes of one variant at a time in the
study data and imputes the masked genotypes by using the remaining study and
reference data. The imputed genotypes are then compared with the original
genotypes to produce the concordance statistics."\\

\BLOCK{ if single_chromosome }
Table~\ref{tab:cross_validation_chr_\VAR{ first_chrom }} shows the
cross-validation results for chromosomes \VAR{ first_chrom }.
\BLOCK{ else }
Tables~\ref{tab:cross_validation_chr_\VAR{ first_chrom }} to
\ref{tab:cross_validation_chr_\VAR{ last_chrom }} show the cross-validation
results for the analyzed chromosomes. Table~\ref{tab:cross_validation} shows
the cross-validation results across the autosomes.
\BLOCK{ endif }

\VAR{ tables }

\pagebreak

